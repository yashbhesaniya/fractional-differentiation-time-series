\documentclass{letter}

% Euler for math | Palatino for rm | Helvetica for ss | Courier for tt
\renewcommand{\rmdefault}{ppl} % rm
\linespread{1.05}        % Palatino needs more leading
\usepackage[scaled]{helvet} % ss
\usepackage{courier} % tt
%\usepackage{euler} % math
\usepackage{eulervm} % a better implementation of the euler package (not in gwTeX)
\normalfont
\usepackage[T1]{fontenc}

\usepackage{mathrsfs}
\usepackage{hyperref}
\usepackage{amssymb,amsthm}
\usepackage{amsmath}
\usepackage{bbold}
\usepackage{color}

% Hyperbolic Functions-----------------------------------------
%
\DeclareMathOperator{\sech}{sech}
\DeclareMathOperator{\csch}{csch}
\DeclareMathOperator{\arcsec}{arcsec}
\DeclareMathOperator{\arccot}{arcCot}
\DeclareMathOperator{\arccsc}{arcCsc}
\DeclareMathOperator{\arccosh}{arcosh}
\DeclareMathOperator{\arcsinh}{arsinh}
\DeclareMathOperator{\arctanh}{artanh}
\DeclareMathOperator{\arcsech}{arcsech}
\DeclareMathOperator{\arccsch}{arcCsch}
\DeclareMathOperator{\arccoth}{arcoth} %


\DeclareMathOperator*{\slim}{s-lim}
\newcommand{\MarkJ}{\marginpar{{\bf J}}}
\newcommand{\MarkC}{\marginpar{{\bf C}}}


\begin{document}

Universidade de S\~ao Paulo \\
Departamento de Matem\'atica Aplicada\\
Christian J\"akel

\bigskip
\centerline{\bf Exercises Chapter 5}

\bigskip
\begin{itemize}

\item[${\bf 5.1}$] Generate a time series from IID Guassian random process. 
This is a memoryless, stationary series: 

\bigskip
We start as always:

\bigskip
\color{blue}
import numpy as np

import pandas as pd

import matplotlib.pyplot as plt

p = print
\color{black}

\bigskip
We compute the weights: 
	\[
		\omega_k = - \omega_{k-1} \frac{d-k+}{k} \; , \qquad k = 0, 1, 2, \ldots \; .    
	\]
This is done in the following code (Snippet 5.1): 

\color{blue}
def getWeights(d, size):
    w=[1.]
    for k in range(1,size):
        w_ = -w[-1]/k*(d-k+1)
        w.append(w_)
    w=np.array(w[::-1]).reshape(-1,1)
    return w


\color{black}

\bigskip
Next we look at the second piece in Snippet 5.1): 

\color{blue}
def plotWeights(dRange, nPlots, size):
    w=pd.DataFrame()
    for d in np.linspace(dRange[0], dRange[1], nPlots):
        w_=getWeights(d, size = size)
        w_ = pd.DataFrame(w_, index = range(w_.shape[0])[:: -1], columns = [d])
        w=w.join(w_, how = "outer")
    ax = w.plot(figsize=(10,8))
    ax.legend(loc='upper left'); plt.show()
    return
 
%%%%%%%%%%

plotWeights(dRange = [0,1], nPlots = 11, size = 6)
plotWeights(dRange = [1,2], nPlots = 11, size = 6)

%%%%%%%%%%

def weight_by_d(dRange=[0,1], nPlots=11, size=6):
    '''
    dRange: the range of d
    nPlots: the number of d we want to check
    size: the data points used as an example
    w: collection of w by different d value
    '''
    
    w=pd.DataFrame()
    
    for d in np.linspace(dRange[0],dRange[1],nPlots):
        w_=getWeights(d,size=size)
        w_=pd.DataFrame(w_,index=range(w_.shape[0])\
        [::-1],columns=[d])
        w=w.join(w_,how='outer')
        
    return w
weight_by_d = weight_by_d()
weight_by_d   
     
%%%%%%%%%%

\begin{itemize}
\item[$a.)$] Compute the ADF Compute the ADF statistic on this series. What is the p-value?
\item[$b.)$] Compute the cumulative sum of the observations. This is a non-stationary
series without memory.
\begin{itemize}
\item[$i.)$] What is the order of integration of this cumulative series?
\item[$ii.)$] Compute the ADF statistic on this series. What is the p-value?
\end{itemize}
\item[$c.)$] Differentiate the series twice. What is the p-value of this over-differentiated
series?
\end{itemize}

\item[${\bf 5.2}$] Generate a time series that follows a sinusoidal function. This is a stationary
series with memory.
\begin{itemize}
\item[$a.)$] Compute the ADF statistic on this series. What is the p-value?
\item[$b.)$] Shift every observation by the same positive value. Compute the cumulative
sum of the observations. This is a non-stationary series with memory.
\begin{itemize}
\item[$i.)$] Compute the ADF statistic on this series. What is the p-value?
\item[$ii.)$] Apply an expanding window fracdiff, with $\tau = 10^{-2}$. 
For what minimum $d$ value do you get a p-value below 5 \%?
\item[$iii.)$] Apply FFD, with $\tau = 10^{-5}$. For what minimum d value do 
you get a p-value below $5\%$?
\end{itemize}
\end{itemize}

\item[${\bf 5.3}$] Take the series from exercise 2.b:
\begin{itemize}
\item[$a.)$] Fit the series to a sine function. What is the R-squared?
\item[$b.)$] Apply FFD(d = 1). Fit the series to a sine function. What is the R-squared?
\item[$c.)$] What value of d maximizes the R-squared of a sinusoidal fit on FFD(d).
Why?
\end{itemize}

\item[${\bf 5.4}$] Take the dollar bar series on E-mini S \& P 500 futures. Using the code
in Snippet 5.3, for some $d \in [0, 2]$, compute $fracDiff_FFD$ ($fracDiff
_FFD(series,d),-d$). What do you get? Why?

\item[${\bf 5.5}$] Take the dollar bar series on E-mini S \&P 500 futures.
\begin{itemize}
\item[$a.)$] Form a new series as a cumulative sum of log-prices.
\item[$b.)$] Apply FFD, with $\tau = 10^{-5}$. Determine for what 
minimum $d \in [0, 2]$ the new series is stationary.
\item[$c.)$] Compute the correlation of the fracdiffseries to the original (untransformed)
series.
\item[$d.)$] Apply an Engel-Granger cointegration test on the original and fracdiffseries.
Are they cointegrated? Why?
\item[$e.)$] Apply a Jarque-Bera normality test on the fracdiff series.
\end{itemize}

\item[${\bf 5.6}$] Take the fracdiff series from exercise 5.
\begin{itemize}
\item[$a.)$] Apply a CUSUM filter (Chapter 2), where h is twice the standard deviation
of the series.
\item[$b.)$] Use the filtered timestamps to sample a features’ matrix. Use as one of the
features the fracdiff value.
\item[$c.)$] Form labels using the triple-barrier method, with symmetric horizontal bar-
riers of twice the daily standard deviation, and a vertical barrier of 5 days.
\item[$d.)$] Fit a bagging classifier of decision trees where:
\begin{itemize}
\item[$i.)$] The observed features are bootstrapped using the 
sequential method from Chapter 4.
\item[$ii.)$] On each bootstrapped sample, sample weights are 
determined using the techniques from Chapter 4.
\end{itemize}
\end{itemize}
\end{itemize}

\end{document}